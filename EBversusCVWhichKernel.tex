\documentclass{amsart}
\usepackage{mathtools,upref,siunitx,upquote,fancyvrb,xspace,color}
\usepackage[hyphens]{url}
\usepackage[utf8]{inputenc}
\usepackage{esdiff}

\DeclareSymbolFont{GreekLetters}{OML}{cmr}{m}{it} %Provide missing letters
\DeclareSymbolFont{UpSfGreekLetters}{U}{cmss}{m}{n} %Provide missing letters
\DeclareMathSymbol{\varrho}{\mathalpha}{GreekLetters}{"25}
\DeclareMathSymbol{\UpSfLambda}{\mathalpha}{UpSfGreekLetters}{"03}
\DeclareMathSymbol{\UpSfSigma}{\mathalpha}{UpSfGreekLetters}{"06}
%\newcommand{\bvec}[1]{\boldsymbol{#1}}
\providecommand{\mathbold}{\boldsymbol}
\newcommand{\bvec}[1]{\mathbold{#1}}
%\newcommand{\bvec}[1]{\text{\boldmath$#1$}}
\newcommand{\avec}[1]{\vec{#1}}
%\renewcommand{\vec}[1] {\text{\boldmath$#1$}}
%\renewcommand{\vec}[1]{\ensuremath{\mathbf{#1}}}
%\newcommand{\vecsym}[1]{\ensuremath{\boldsymbol{#1}}}
\newcommand{\vecsym}[1]{\ensuremath{\mathbold{#1}}}
\def\bbl{\text{\boldmath$\{$}}
\def\bbr{\text{\boldmath$\}$}}
\newcommand{\bbrace}[1]{\bbl #1 \bbr}
\newcommand{\bbbrace}[1]{\mathopen{\pmb{\bigg\{}}#1\mathclose{\pmb{\bigg\}}}}
\def\betahat{\hat\beta}
\newcommand{\dif}{{\rm d}}

\newlength{\overwdth}
\def\overstrike#1{ 
\settowidth{\overwdth}{#1}\makebox[0pt][l]{\rule[0.5ex]{\overwdth}{0.1ex}}#1}

\makeatletter
\newcommand*\bigcdot{\mathpalette\bigcdot@{.7}}
\newcommand*\bigcdot@[2]{\mathbin{\vcenter{\hbox{\scalebox{#2}{$\m@th#1\bullet$}}}}}
\makeatother

\def\abs#1{\ensuremath{\left \lvert #1 \right \rvert}}
\newcommand{\normabs}[1]{\ensuremath{\lvert #1 \rvert}}
\newcommand{\bigabs}[1]{\ensuremath{\bigl \lvert #1 \bigr \rvert}}
\newcommand{\Bigabs}[1]{\ensuremath{\Bigl \lvert #1 \Bigr \rvert}}
\newcommand{\biggabs}[1]{\ensuremath{\biggl \lvert #1 \biggr \rvert}}
\newcommand{\Biggabs}[1]{\ensuremath{\Biggl \lvert #1 \Biggr \rvert}}
\newcommand{\norm}[2][{}]{\ensuremath{\left \lVert #2 \right \rVert}_{#1}}
\newcommand{\normnorm}[2][{}]{\ensuremath{\lVert #2 \rVert}_{#1}}
\newcommand{\bignorm}[2][{}]{\ensuremath{\bigl \lVert #2 \bigr \rVert}_{#1}}
\newcommand{\Bignorm}[2][{}]{\ensuremath{\Bigl \lVert #2 \Bigr \rVert}_{#1}}
\newcommand{\biggnorm}[2][{}]{\ensuremath{\biggl \lVert #2 \biggr \rVert}_{#1}}
\newcommand{\Biggnorm}[2][{}]{\ensuremath{\Biggl \lVert #2 \Biggr \rVert}_{#1}}
\newcommand{\ip}[3][{}]{\ensuremath{\left \langle #2, #3 \right \rangle_{#1}}}

\newcommand{\bigvecpar}[3]{\ensuremath{\bigl ( #1 \bigr )_{#2}^{#3}}}
\newcommand{\Bigvecpar}[3]{\ensuremath{\Bigl ( #1 \Bigr )_{#2}^{#3}}}
\newcommand{\biggvecpar}[3]{\ensuremath{\biggl ( #1 \biggr )_{#2}^{#3}}}
\newcommand{\bigpar}[1]{\ensuremath{\bigl ( #1 \bigr )}}
\newcommand{\Bigpar}[1]{\ensuremath{\Bigl ( #1 \Bigr )}}
\newcommand{\biggpar}[1]{\ensuremath{\biggl ( #1 \biggr )}}

\newcommand{\IIDsim}{\overset{\textup{IID}}{\sim}}
\newcommand{\LDsim}{\overset{\textup{LD}}{\sim}}

\DeclareMathOperator{\success}{succ}
\DeclareMathOperator{\sinc}{sinc}
\DeclareMathOperator{\sech}{sech}
\DeclareMathOperator{\csch}{csch}
\DeclareMathOperator{\dist}{dist}
\DeclareMathOperator{\spn}{span}
\DeclareMathOperator{\sgn}{sgn}
\DeclareMathOperator*{\rmse}{rmse}
\DeclareMathOperator{\Prob}{\mathbb{P}}
\DeclareMathOperator{\Ex}{\mathbb{E}}
\DeclareMathOperator{\rank}{rank}
\DeclareMathOperator{\erfc}{erfc}
\DeclareMathOperator{\erf}{erf}
\DeclareMathOperator{\cov}{cov}
\DeclareMathOperator{\cost}{cost}
\DeclareMathOperator{\comp}{comp}
\DeclareMathOperator{\corr}{corr}
\DeclareMathOperator{\diag}{diag}
\DeclareMathOperator{\var}{var}
\DeclareMathOperator{\opt}{opt}
\DeclareMathOperator{\brandnew}{new}
\DeclareMathOperator{\std}{std}
\DeclareMathOperator{\kurt}{kurt}
\DeclareMathOperator{\med}{med}
\DeclareMathOperator{\vol}{vol}
\DeclareMathOperator{\bias}{bias}
\DeclareMathOperator*{\argmax}{argmax}
\DeclareMathOperator*{\argmin}{argmin}
\DeclareMathOperator{\sign}{sign}
\DeclareMathOperator{\spann}{span}
\DeclareMathOperator{\cond}{cond}
\DeclareMathOperator{\trace}{trace}
\DeclareMathOperator{\Si}{Si}
%\DeclareMathOperator{\diag}{diag}
\DeclareMathOperator{\col}{col}
\DeclareMathOperator{\nullspace}{null}
\DeclareMathOperator{\Order}{{\mathcal O}}
%\DeclareMathOperator{\rank}{rank}

\newcommand{\vzero}{\bvec{0}}
\newcommand{\vone}{\bvec{1}}
\newcommand{\vinf}{\bvec{\infty}}
\newcommand{\va}{\bvec{a}}
\newcommand{\vA}{\bvec{A}}
\newcommand{\vb}{\bvec{b}}
\newcommand{\vB}{\bvec{B}}
\newcommand{\vc}{\bvec{c}}
\newcommand{\vC}{\bvec{C}}
\newcommand{\vd}{\bvec{d}}
\newcommand{\vD}{\bvec{D}}
\newcommand{\ve}{\bvec{e}}
\newcommand{\vf}{\bvec{f}}
\newcommand{\vF}{\bvec{F}}
\newcommand{\vg}{\bvec{g}}
\newcommand{\vG}{\bvec{G}}
\newcommand{\vh}{\bvec{h}}
\newcommand{\vH}{\bvec{H}}
\newcommand{\vi}{\bvec{i}}
\newcommand{\vj}{\bvec{j}}
\newcommand{\vk}{\bvec{k}}
\newcommand{\vK}{\bvec{K}}
\newcommand{\vl}{\bvec{l}}
\newcommand{\vell}{\bvec{\ell}}
\newcommand{\vL}{\bvec{L}}
\newcommand{\vm}{\bvec{m}}
\newcommand{\vp}{\bvec{p}}
\newcommand{\vq}{\bvec{q}}
\newcommand{\vr}{\bvec{r}}
\newcommand{\vs}{\bvec{s}}
\newcommand{\vS}{\bvec{S}}
\newcommand{\vt}{\bvec{t}}
\newcommand{\vT}{\bvec{T}}
\newcommand{\vu}{\bvec{u}}
\newcommand{\vU}{\bvec{U}}
\newcommand{\vv}{\bvec{v}}
\newcommand{\vV}{\bvec{V}}
\newcommand{\vw}{\bvec{w}}
\newcommand{\vW}{\bvec{W}}
\newcommand{\vx}{\bvec{x}}
\newcommand{\vX}{\bvec{X}}
\newcommand{\vy}{\bvec{y}}
\newcommand{\vY}{\bvec{Y}}
\newcommand{\vz}{\bvec{z}}
\newcommand{\vZ}{\bvec{Z}}

\newcommand{\ai}{\avec{\imath}}
\newcommand{\ak}{\avec{k}}
\newcommand{\avi}{\avec{\bvec{\imath}}}
\newcommand{\at}{\avec{t}}
\newcommand{\avt}{\avec{\vt}}
\newcommand{\ax}{\avec{x}}
\newcommand{\ah}{\avec{h}}
\newcommand{\akappa}{\avec{\kappa}}
\newcommand{\avx}{\avec{\vx}}
\newcommand{\ay}{\avec{y}}
\newcommand{\avy}{\avec{\vy}}
\newcommand{\avz}{\avec{\vz}}
\newcommand{\avzero}{\avec{\vzero}}
\newcommand{\aomega}{\avec{\omega}}
\newcommand{\avomega}{\avec{\vomega}}
\newcommand{\anu}{\avec{\nu}}
\newcommand{\avnu}{\avec{\vnu}}
\newcommand{\aDelta}{\avec{\Delta}}
\newcommand{\avDelta}{\avec{\vDelta}}

\newcommand{\valpha}{\bvec{\alpha}}
\newcommand{\vbeta}{\bvec{\beta}}
\newcommand{\vgamma}{\bvec{\gamma}}
\newcommand{\vGamma}{\bvec{\Gamma}}
\newcommand{\vdelta}{\bvec{\delta}}
\newcommand{\vDelta}{\bvec{\Delta}}
\newcommand{\vphi}{\bvec{\phi}}
\newcommand{\vvphi}{\bvec{\varphi}}
\newcommand{\vPhi}{\bvec{\Phi}}
\newcommand{\vomega}{\bvec{\omega}}
\newcommand{\vkappa}{\bvec{\kappa}}
\newcommand{\vlambda}{\bvec{\lambda}}
\newcommand{\vmu}{\bvec{\mu}}
\newcommand{\vnu}{\bvec{\nu}}
\newcommand{\vpsi}{\bvec{\psi}}
\newcommand{\vPsi}{\bvec{\Psi}}
\newcommand{\vepsilon}{\bvec{\epsilon}}
\newcommand{\veps}{\bvec{\varepsilon}}
\newcommand{\veta}{\bvec{\eta}}
\newcommand{\vxi}{\bvec{\xi}}
\newcommand{\vtheta}{\bvec{\theta}}
\newcommand{\vtau}{\bvec{\tau}}
\newcommand{\vzeta}{\bvec{\zeta}}

\newcommand{\hA}{\widehat{A}}
\newcommand{\hvb}{\hat{\vb}}
\newcommand{\hcc}{\widehat{\cc}}
\newcommand{\hD}{\widehat{D}}
\newcommand{\hE}{\widehat{E}}
\newcommand{\hf}{\widehat{f}}
\newcommand{\hF}{\widehat{F}}
\newcommand{\hg}{\hat{g}}
\newcommand{\hvf}{\widehat{\bvec{f}}}
\newcommand{\hh}{\hat{h}}
\newcommand{\hH}{\widehat{H}}
\newcommand{\hi}{\hat{\imath}}
\newcommand{\hI}{\hat{I}}
\newcommand{\hci}{\widehat{\ci}}
\newcommand{\hj}{\hat{\jmath}}
\newcommand{\hJ}{\widehat{J}}
\newcommand{\hvL}{\widehat{\bvec{L}}}
\newcommand{\hp}{\hat{p}}
\newcommand{\hP}{\widehat{P}}
\newcommand{\hS}{\widehat{S}}
\newcommand{\hv}{\hat{v}}
\newcommand{\hV}{\widehat{V}}
\newcommand{\hx}{\hat{x}}
\newcommand{\hX}{\widehat{X}}
\newcommand{\hvX}{\widehat{\vX}}
\newcommand{\hy}{\hat{y}}
\newcommand{\hvy}{\hat{\vy}}
\newcommand{\hY}{\widehat{Y}}
\newcommand{\hvY}{\widehat{\vY}}
\newcommand{\hZ}{\widehat{Z}}
\newcommand{\hvZ}{\widehat{\vZ}}

\newcommand{\halpha}{\hat{\alpha}}
\newcommand{\hvalpha}{\bvec{\widehat{\alpha}}}
\newcommand{\hbeta}{\hat{\beta}}
\newcommand{\hvbeta}{\hat{\vbeta}}
\newcommand{\hgamma}{\hat{\gamma}}
\newcommand{\hvgamma}{\hat{\vgamma}}
\newcommand{\hdelta}{\hat{\delta}}
\newcommand{\hvareps}{\hat{\varepsilon}}
\newcommand{\hveps}{\hat{\veps}}
\newcommand{\hmu}{\hat{\mu}}
\newcommand{\hnu}{\hat{\nu}}
\newcommand{\hvnu}{\widehat{\vnu}}
\newcommand{\homega}{\widehat{\omega}}
\newcommand{\hPi}{\widehat{\Pi}}
\newcommand{\hrho}{\hat{\rho}}
\newcommand{\hsigma}{\hat{\sigma}}
\newcommand{\htheta}{\hat{\theta}}
\newcommand{\hvtheta}{\bvec{\widehat{\theta}}}
\newcommand{\hTheta}{\hat{\Theta}}
\newcommand{\htau}{\hat{\tau}}
\newcommand{\hxi}{\hat{\xi}}
\newcommand{\hvxi}{\hat{\vxi}}

\newcommand{\otau}{\overline{\tau}}
\newcommand{\oY}{\overline{Y}}

\newcommand{\rD}{\mathring{D}}
\newcommand{\rf}{\mathring{f}}
\newcommand{\rV}{\mathring{V}}

\newcommand{\ta}{\tilde{a}}
\newcommand{\tA}{\tilde{A}}
\newcommand{\tmA}{\widetilde{\mA}}
\newcommand{\tvb}{\widetilde{\vb}}
\newcommand{\tcb}{\widetilde{\cb}}
\newcommand{\tB}{\widetilde{B}}
\newcommand{\tc}{\tilde{c}}
\newcommand{\tvc}{\tilde{\vc}}
\newcommand{\tfc}{\tilde{\fc}}
\newcommand{\tC}{\widetilde{C}}
\newcommand{\tcc}{\widetilde{\cc}}
\newcommand{\tD}{\widetilde{D}}
\newcommand{\te}{\tilde{e}}
\newcommand{\tE}{\widetilde{E}}
\newcommand{\tf}{\widetilde{f}}
\newcommand{\tF}{\widetilde{F}}
\newcommand{\tvf}{\tilde{\vf}}
\newcommand{\tcf}{\widetilde{\cf}}
\newcommand{\tg}{\tilde{g}}
\newcommand{\tvg}{\widetilde{\vg}}
\newcommand{\tG}{\widetilde{G}}
\newcommand{\tildeh}{\tilde{h}}
\newcommand{\tH}{\widetilde{H}}
\newcommand{\tch}{\widetilde{\ch}}
\newcommand{\tK}{\widetilde{K}}
\newcommand{\tvk}{\tilde{\vk}}
\newcommand{\tM}{\widetilde{M}}
\newcommand{\tn}{\tilde{n}}
\newcommand{\tN}{\widetilde{N}}
\newcommand{\tQ}{\widetilde{Q}}
\newcommand{\tR}{\widetilde{R}}
\newcommand{\tS}{\widetilde{S}}
\newcommand{\tvS}{\widetilde{\vS}}
\newcommand{\tT}{\widetilde{T}}
\newcommand{\tv}{\tilde{v}}
\newcommand{\tV}{\widetilde{V}}
\newcommand{\tvx}{\tilde{\vx}}
\newcommand{\tW}{\widetilde{W}}
\newcommand{\tx}{\tilde{x}}
\newcommand{\tX}{\widetilde{X}}
\newcommand{\tvX}{\widetilde{\vX}}
\newcommand{\ty}{\tilde{y}}
\newcommand{\tvy}{\tilde{\vy}}
\newcommand{\tz}{\tilde{z}}
\newcommand{\tZ}{\widetilde{Z}}
\newcommand{\tL}{\widetilde{L}}
\newcommand{\tP}{\widetilde{P}}
\newcommand{\tY}{\widetilde{Y}}
\newcommand{\tmH}{\widetilde{\mH}}
\newcommand{\tmK}{\widetilde{\mK}}
\newcommand{\tmM}{\widetilde{\mM}}
\newcommand{\tmQ}{\widetilde{\mQ}}
\newcommand{\tct}{\widetilde{\ct}}
\newcommand{\talpha}{\tilde{\alpha}}
\newcommand{\tdelta}{\tilde{\delta}}
\newcommand{\tDelta}{\tilde{\Delta}}
\newcommand{\tvareps}{\tilde{\varepsilon}}
\newcommand{\tveps}{\tilde{\veps}}
\newcommand{\tlambda}{\tilde{\lambda}}
\newcommand{\tmu}{\tilde{\mu}}
\newcommand{\tnu}{\tilde{\nu}}
\newcommand{\trho}{\tilde{\rho}}
\newcommand{\tvarrho}{\tilde{\varrho}}
\newcommand{\ttheta}{\tilde{\theta}}
\newcommand{\tsigma}{\tilde{\sigma}}
\newcommand{\tvmu}{\tilde{\vmu}}
\newcommand{\tphi}{\tilde{\phi}}
\newcommand{\tPhi}{\widetilde{\Phi}}
\newcommand{\tvphi}{\tilde{\vphi}}
\newcommand{\ttau}{\tilde{\tau}}
\newcommand{\txi}{\tilde{\xi}}
\newcommand{\tvxi}{\tilde{\vxi}}


\newcommand{\mA}{\mathsf{A}}
\newcommand{\mB}{\mathsf{B}}
\newcommand{\mC}{\mathsf{C}}
\newcommand{\vmC}{\bvec{\mC}}
\newcommand{\mD}{\mathsf{D}}
\newcommand{\mF}{\mathsf{F}}
\newcommand{\mG}{\mathsf{G}}
\newcommand{\mH}{\mathsf{H}}
\newcommand{\mI}{\mathsf{I}}
\newcommand{\mK}{\mathsf{K}}
\newcommand{\mL}{\mathsf{L}}
\newcommand{\mM}{\mathsf{M}}
\newcommand{\mP}{\mathsf{P}}
\newcommand{\mQ}{\mathsf{Q}}
\newcommand{\mR}{\mathsf{R}}
\newcommand{\mS}{\mathsf{S}}
\newcommand{\mT}{\mathsf{T}}
\newcommand{\mU}{\mathsf{U}}
\newcommand{\mV}{\mathsf{V}}
\newcommand{\mW}{\mathsf{W}}
\newcommand{\mX}{\mathsf{X}}
\newcommand{\mY}{\mathsf{Y}}
\newcommand{\mZ}{\mathsf{Z}}
\newcommand{\mLambda}{\UpSfLambda}
\newcommand{\mSigma}{\UpSfSigma}
\newcommand{\mzero}{\mathsf{0}}
\newcommand{\mGamma}{\mathsf{\Gamma}}

\newcommand{\bbE}{\mathbb{E}}
\newcommand{\bbF}{\mathbb{F}}
\newcommand{\bbK}{\mathbb{K}}
\newcommand{\bbV}{\mathbb{V}}
\newcommand{\bbZ}{\mathbb{Z}}
\newcommand{\bbone}{\mathbbm{1}}
\newcommand{\naturals}{\mathbb{N}}
\newcommand{\reals}{\mathbb{R}}
\newcommand{\integers}{\mathbb{Z}}
\newcommand{\natzero}{\mathbb{N}_{0}}
\newcommand{\rationals}{\mathbb{Q}}
\newcommand{\complex}{\mathbb{C}}

\newcommand{\ca}{\mathcal{A}}
\newcommand{\cb}{\mathcal{B}}
\providecommand{\cc}{\mathcal{C}}
\newcommand{\cd}{\mathcal{D}}
\newcommand{\ce}{\mathcal{E}}
\newcommand{\cf}{\mathcal{F}}
\newcommand{\cg}{\mathcal{G}}
\newcommand{\ch}{\mathcal{H}}
\newcommand{\ci}{\mathcal{I}}
\newcommand{\cj}{\mathcal{J}}
\newcommand{\ck}{\mathcal{K}}
\newcommand{\cl}{\mathcal{L}}
\newcommand{\cm}{\mathcal{M}}
\newcommand{\tcm}{\widetilde{\cm}}
\newcommand{\cn}{\mathcal{N}}
\newcommand{\cp}{\mathcal{P}}
\newcommand{\cq}{\mathcal{Q}}
\newcommand{\calr}{\mathcal{R}}
\newcommand{\cs}{\mathcal{S}}
\newcommand{\ct}{\mathcal{T}}
\newcommand{\cu}{\mathcal{U}}
\newcommand{\cv}{\mathcal{V}}
\newcommand{\cw}{\mathcal{W}}
\newcommand{\cx}{\mathcal{X}}
\newcommand{\tcx}{\widetilde{\cx}}
\newcommand{\cy}{\mathcal{Y}}
\newcommand{\cz}{\mathcal{Z}}

\newcommand{\fc}{\mathfrak{c}}
\newcommand{\fC}{\mathfrak{C}}
\newcommand{\fh}{\mathfrak{h}}
\newcommand{\fu}{\mathfrak{u}}

\newcommand{\me}{\ensuremath{\mathrm{e}}} % for math number 'e', 2.718 281 8..., tha base of natural logarithms
\newcommand{\mi}{\ensuremath{\mathrm{i}}} % for math number 'i', the imaginary unit
\newcommand{\mpi}{\ensuremath{\mathrm{\pi}}} % for math number 'pi', the circumference of a circle of diameter 1


\newcommand{\rvf}{\mathring{\vf}}
\newcommand{\romega}{\mathring{\omega}}
\newcommand{\gp}{\mathcal{G}\!\mathcal{P}}


\usepackage{algpseudocode}
\usepackage{algorithm, algorithmicx}
\algnewcommand\algorithmicparam{\textbf{Parameters:}}
\algnewcommand\PARAM{\item[\algorithmicparam]}
\algnewcommand\algorithmicinput{\textbf{Input:}}
\algnewcommand\INPUT{\item[\algorithmicinput]}
\algnewcommand\RETURN{\State \textbf{Return }}



\begin{document}
\title{Sample Document}
\author{Fred J. Hickernell}
\author{Houman Owhadi}
\author{Aleksei Sorokin}
\begin{abstract}This project is where all of the files and commands go that are needed elsewhere
\end{abstract}

\maketitle

\section{Introduction}

\section{Background}
Let $f$ have an absolutely summable Fourier series:
\begin{equation*}
    f(\vx) = \sum_{\vk \in \integers^d} \hf(\vk) \exp(2 \pi \sqrt{-1} \vk^T \vx), \qquad \hf(\vk) : = \int_{[0,1]^d} f(\vx)  \exp(-2 \pi \sqrt{-1} \vk^T \vx) \, \dif \vx.
\end{equation*}
Suppose that the sampling sites or nodes constitutes $\{ \vx_1, \ldots, \vx_n\}$ is a shifted integration lattice, 
\begin{equation} \label{eq:shift_lat}
    \vx_i = \vh i /n + \vDelta \pmod \vone, \qquad i = 0, \ldots, n - 1
\end{equation}
where $\vh \in \{1, \ldots, n-1\}^d$.  Then the discrete Fourier transform (DFT) of the function data, $\{ f(\vx_0), \ldots, f(\vx_{n-1})\}$, is given by 
\begin{align}
\nonumber
    \tf(\vk) &= \frac 1n \sum_{i=1}^n f(\vx_i) \exp(-2 \pi \sqrt{-1} \vk^T \vx_i) \\
\nonumber
    & = \frac 1n \sum_{i=1}^n f(\vx_i) \exp(-2 \pi \sqrt{-1} [\vk^T\vh i/n + \vk^T\vDelta]) \\
\nonumber
    & = \rf(\vk^T \vh \bmod n) \exp(-2 \pi \sqrt{-1} \vk^T\vDelta), \\
    \text{where } 
    \rf(j) &: = \frac 1n \sum_{i=1}^n f(\vx_i) \exp(-2 \pi \sqrt{-1} ij/n), \quad  j = 0, \ldots, n-1 \qquad \label{eq:fring}\\
\nonumber
    & = \frac 1n \sum_{i=1}^n \sum_{\vk \in \integers^d} \hf(\vk) \exp(2 \pi \sqrt{-1} \vk^T \vx_i) \exp(-2 \pi \sqrt{-1} ij/n) \\
\nonumber
    & = \frac 1n \sum_{\vk \in \integers^d} \hf(\vk) \sum_{i=1}^n \exp(2 \pi \sqrt{-1} i (\vk^T \vh - j) /n + \vk^T \vDelta) \\
\nonumber
    & = \sum_{\substack{\vk \in \integers^d \\ \vk^T \vh \bmod n = j}} \hf(\vk) \exp(2 \pi \sqrt{-1} \vk^T \vDelta) \\
\end{align}
Note that the DFT takes on only $n$ distinct values. 

We want to consider functions that are samples from a Gaussian process with a covariance kernel, $K$, and/or elements in a Hilbert space with reproducing kernel $K$.  The $K$ is assumed to be  periodic:
\begin{equation} \label{eq:Kseries}
    K(\vt,\vx) = \sum_{\vk \in \integers^d} \omega_{\vk}^2 \exp(2 \pi \sqrt{-1} \vk^T (\vt - \vx))
\end{equation}
where the set of weights, $\vomega : = (\omega_{\vk})_{\vk \in \integers^d}$, satisfies the following summability condition:
\begin{equation*}
    \sum_{\vk \in \integers^d} \omega_{\vk}^2 < \infty.
\end{equation*}
We must assume that $\omega_{-\vk} = \omega_{\vk}$  because the kernel $K$ must be symmetric in its arguments. 

For such a kernel, the inner product of the associated reproducing kernel Hilbert space is defined by 
\begin{equation*}
    \ip[\vomega]{f}{g} = \sum_{\vk \in \integers^d} 
    \frac{\overline{\hf(\vk)} \hg(\vk)}{\omega_{\vk}^2}
\end{equation*}
The reproducing property follows because $K(\cdot,\vx)$ is in the Hilbert space, and 
\begin{equation*}
    \ip[\vomega]{K(\cdot,\vx)}{f} 
    = \sum_{\vk \in \integers^d} 
    \frac{ \omega_{\vk}^2 \exp(2 \pi \sqrt{-1} \vk^T \vx)) \hf(\vk)}{\omega_{\vk}^2} = f(\vx).
\end{equation*}

Now consider the entries of the Gram matrix for a periodic kernel where the sampling sites or nodes are a shifted lattice as defined in \eqref{eq:shift_lat}:
\begin{align}
\nonumber
    K(\vx_i,\vx_j)  &= \sum_{\vk \in \integers^d} \omega_{\vk}^2 \exp(2 \pi \sqrt{-1} \vk^T(\vx_i - \vx_j)) \\
    \nonumber
    & = \sum_{\vk \in \integers^d} \omega_{\vk}^2 \exp(2 \pi \sqrt{-1} \vk^T \vh (i-j)/n) \\
    \label{eq:lambdaomega}
    & = \frac 1n \sum_{l = 0}^{n-1} \lambda_l \exp(2 \pi \sqrt{-1} l (i-j)/n), \quad \lambda_l =: n\sum_{\substack{\vk \in \integers^d \\ \vk^T\vh \bmod n = l}} \omega_{\vk}^2, \\
    \nonumber
    \MoveEqLeft \underbrace{\bigl ( K(\vx_i,\vx_j)\bigr)_{i,j = 0}^{n-1}}_{\mK} \\
    \nonumber
    &
    = \frac 1n \underbrace{\bigl ( \exp(2 \pi \sqrt{-1} il/n) \bigr)_{i,l = 0}^{n-1}}_{\mV} 
    \underbrace{\bigl ( \lambda_l \delta_{l,m} \bigr)_{l,m = 0}^{n-1}}_{\mLambda = \diag(\vlambda)} 
    \underbrace{\bigl ( \exp(- 2 \pi \sqrt{-1} mj/n) \bigr)_{m,j = 0}^{n-1}}_{\mV^H}
\end{align}

Since $\mV^H \mV = n \mI$, it follows that $\mV^{-1} = \mV^H/n$.  Then, 
$\mK \mV  =\mV \mLambda$, and so $\vlambda$ are the eigenvalues of $\mK$.  It also follows that $\mK^{-1} = n^{-1} \mV \mLambda \mV^H$.  
Letting $\vf := \bigl( f(\vx_i) \bigr)_{i=1}^n$ and $\rvf := \bigl( \rf(\vx_i) \bigr)_{i=1}^n$, it follows from \eqref{eq:fring} that the DFT and the inverse DFT (IDFT) can be expressed in terms of matrix operations involving $\mV^H$ and $\mV$
\begin{equation}
    \rvf = \frac 1n \mV^H \vf, \qquad \vf = \mV \rvf.
\end{equation}

Note that the first (or zeroth) column of $\mK$ is 
\begin{equation*}
    \vK_0 = \bigl ( K(\vx_i,\vx_0)\bigr)_{i = 0}^{n-1}
    = \biggl( \frac 1n \sum_{l = 0}^{n-1} \lambda_l \exp(2 \pi \sqrt{-1} l i/n)\biggr)_{i=1}^n = \frac 1n \mV \vlambda.
\end{equation*}
Thus, the eigenvalues of the Gram matrix are simply the $n$ times the DFT of its first column:
\begin{equation*}
    \mV^{H} \vK_0 = \mV^{H} \frac 1n \mV \vlambda = \vlambda.
\end{equation*}


\section{Optimizing the Kernel According to Empirical Bayes}
\subsection{Initial optimization for the vertical scale factor}
We assume that the function $f$ is an instance of a zero mean Gaussian process, i.e., $f \in \gp(0,s^2K_{\vtheta})$, where we now show the explicit dependence of the
kernel $K$ on the hyperparameters $\vtheta$ as well as the hyperparameter $s^2$, which is the overall vertical scale.
This means that $\{\omega_{\vk}\}_{\vk \in \integers^d}$, and $\{\lambda_l \}_{l=0}^{n-1}$ all depend on $\vtheta$ as well, but $\mV$ does not.

In the empirical Bayes setting, we assume that $f \in \gp(0,s^2K_{\vtheta})$, where we now show the explicit dependence of the
kernel $K$ on the hyperparameters $\vtheta$ as well as the hyperparameter $s^2$, which is the overall vertical scaling.  
The log likelihood (ignoring additive constants) given the data $\vf$ is written as 
\begin{align}
	\text{loglike}(s^2,\vtheta) 
	& = - \frac 1{2} \vf^T (s^2\mK_{\vtheta})^{-1}\vf - \frac 12 \log\bigl(\det(s^2 \mK_{\vtheta})\bigr)  \\
	\nonumber 
	& = - \frac 1{2 s^2} \vf^T\Bigl(\frac 1n \mV \mLambda_{\vtheta}^{-1} \mV^H \Bigr)\vf 
	- \frac 12 \log\biggl(\prod_{i=0}^{n-1} \lambda_{\vtheta,i}\biggr) - n \log(s)  \\
	\nonumber 
	& = - \frac n{2 s^2} \rvf^T\mLambda_{\vtheta}^{-1} \rvf 
- \frac 12 \sum_{i=0}^{n-1} \log(\lambda_{\vtheta,i}) - \frac{n}2 \log(s^2).
\end{align}

Maximizing the above quantity with respect to $s^2$ leads to 
\begin{equation*}
    0  =\frac{\partial \text{loglike}(s^2,\vtheta)}{\partial s^2}
    = \frac n{2 s^4} \rvf^T\mLambda_{\vtheta}^{-1} \rvf 
 - \frac{n}2 \frac {1}{s^2}
 \iff s^2 = s_{\text{EB}}^2 := \rvf^T\mLambda_{\vtheta}^{-1} \rvf .
\end{equation*}
Then, ignoring additive constants,
\begin{equation*}
    \text{loglike}(s_{\text{EB}}^2,\vtheta) 
	= 
- \frac 12 \sum_{i=0}^{n-1} \log(\lambda_{\vtheta,i}) - \frac n2 \log\bigl(\rvf^T\mLambda_{\vtheta}^{-1} \rvf \bigr).
\end{equation*}
This means that the empirical Bayes formula for $\vtheta$ is
\begin{equation} \label{eq:EBOpt}
    \vtheta_{\text{EB}} 
    = \argmin_{\vtheta} \left [\log\left( \sum_{i=0}^{n-1} \frac{\lvert\rf_i\rvert^2}{\lambda_{\vtheta,i}} \right)
    + \frac 1n \sum_{i=0}^{n-1} \log(\lambda_{\vtheta,i} ) \right ]
\end{equation}


\subsection{Optimizing  $\vtheta$ assuming arbitrary $\lambda_{\vtheta,i}$}
Suppose for a moment that the $\lambda_{\vtheta,i}$ can be chosen arbitrarily.   We know this is not true, but it is easy to solve the problem under this assumption.  Dropping the $\vtheta$ dependence, we want to satisfy
\begin{multline*} 
    - \frac{\lvert\rf_i\rvert^2}{\lambda_{\vtheta,i}^2} \left( \sum_{j=0}^{n-1} \frac{\lvert\rf_j\rvert^2}{\lambda_{\vtheta, j}} \right)^{-1} + \frac 1{n\lambda_{\vtheta,i}} = 0, \quad i = 0, \ldots, n-1
    \\
    \iff
    \frac{\lvert\rf_i\rvert^2}{\lambda_{\vtheta,i}} = \frac 1n \sum_{j=0}^{n-1} \frac{\lvert\rf_j\rvert^2}{\lambda_{\vtheta,j}}, \quad i = 0, \ldots, n-1.
\end{multline*}
Since the right side of this equation is independent of $i$, it follows that $\lambda_{\vtheta,i} = c \lvert\rf_i\rvert^2$ for some constant $c$ and $i = 0, \ldots, n-1$.  Plugging this into the equation above determines $c$ to be unity.

\subsection{Optimizing  $\vtheta$ assuming $\omega_{\theta,\vk} = \romega_{\vk}^{\theta}$}
Suppose that the scalar parameter $\theta$ appears only in the power of the eigenvalues of the kernel, i.e., $\omega_{\theta,\vk} = \romega_{\vk}^{\theta}$.  In this case, $\theta$ is like a smoothness parameter. By \eqref{eq:lambdaomega} it follows that
\begin{equation}
    \lambda_{\theta,i} = n\sum_{\substack{\vk \in \integers^d \\ \vk^T\vh \bmod n = l}} \romega_{\vk}^{2\theta}, \qquad 
    \frac{\dif \lambda_{\theta,i}}{\dif \theta} = 2\theta n\sum_{\substack{\vk \in \integers^d \\ \vk^T\vh \bmod n = l}} \romega_{\vk}^{2\theta-1}
\end{equation}

Using these formulas to perform the optimization in \eqref{eq:EBOpt}, it follows that
\begin{align*} 
   0 & 
   = - \left(\sum_{i=0}^{n-1}\frac{\lvert\rf_i\rvert^2} 
   {\lambda_{\vtheta,i}^2}  \frac{\dif \lambda_{\theta,i}}{\dif \theta} \right)
   \left( \sum_{j=0}^{n-1} \frac{\lvert\rf_j\rvert^2}{\lambda_{\vtheta, j}} \right)^{-1} 
   + \sum_{i=0}^{n-1} \frac 1{n\lambda_{\vtheta,i}} \frac{\dif \lambda_{\theta,i}}{\dif \theta}
    \\
    & \iff
    \frac{\lvert\rf_i\rvert^2}{\lambda_{\vtheta,i}} = \frac 1n \sum_{j=0}^{n-1} \frac{\lvert\rf_j\rvert^2}{\lambda_{\vtheta,j}}, \quad i = 0, \ldots, n-1.
\end{align*}

\section{Kernel Flow Estimator} 

Following \cite{chen2021consistency}, we are interested in optimizing the loss 
$$L_\mathrm{KF} = 1- \frac{\vf_{:n/2}^T \left(\mK_{:n/2,:n/2}\right)^{-1} \vf_{:n/2}}{\vf^T \mK^{-1} \vf}.$$
Note that this is invariant of a potential scaling factor $s^2$, so we ignore it. For lattice points in \emph{natural order} (not the linear ordering assumed throughout the rest of this article), we may write 
$$\vlambda = n \begin{pmatrix} \mV^H & \mW \mV^H \\ \mV^H & - \mW \mV^H \end{pmatrix} \begin{pmatrix} \vk_{:n/2} \\ \vk_{n/2:} \end{pmatrix} = n \begin{pmatrix} \mV^H \vk_{:n/2} + \mW \mV^H \vk_{n/2:} \\ \mV^H \vk_{:n/2} - \mW \mV^H \vk_{n/2:} \end{pmatrix}$$
where now $\mV$ is of size $n/2 \times n/2$ and $\mW$ is diagonal. Therefore, 
$$\frac{1}{4} \left(\vlambda_{:n/2} + \vlambda_{n/2:}\right) = \frac{n}{2} \mV^H \vk_{:n/2}.$$
Similarly, 
$$\frac{1}{2} \left(\rvf_{:n/2} + \rvf_{n/2:}\right) = \mV^H \vf_{:n/2}.$$
Therefore,
\begin{align*}
    \vf_{:n/2}^T \mK_{:n/2,:n/2}^{-1} \vf_{:n/2} 
    &= (\mV^H \vf_{:n/2})^H \diag\left(\frac{n}{2}\mV^H \vk_{:n/2}\right)^{-1} (\mV^H \vf_{:n/2}) \\
    &= \left(\rvf_{:n/2}+\rvf_{n/2:}\right)^H \left(\mLambda_{:n/2,:n/2}+\mLambda_{n/2:,n/2:}\right)^{-1}\left(\rvf_{:n/2}+\rvf_{n/2:}\right) \\
    &= \sum_{i=0}^{n/2-1} \frac{\lvert \rf_i + \rf_{n/2+i} \rvert^2}{\lambda_{\vtheta,i} + \lambda_{\vtheta,n/2+i}}
\end{align*}
and 
$$\vf^T \mK^{-1} \vf = \rvf^H \mLambda^{-1} \rvf = \sum_{i=0}^{n/2-1} \left(\frac{\lvert \rf_i \rvert^2}{\lambda_{\vtheta,i}} + \frac{\lvert \rf_{n/2+i} \rvert^2}{\lambda_{\vtheta,n/2+i}} \right)$$
Therefore, 
\begin{align*}
    L_\mathrm{KF} &= 1 - \left[\sum_{i=0}^{n/2-1} \frac{\lvert \rf_i + \rf_{n/2+i} \rvert^2}{\lambda_{\vtheta,i} + \lambda_{\vtheta,n/2+i}}\right] \left[\sum_{i=0}^{n/2-1} \left(\frac{\lvert \rf_i \rvert^2}{\lambda_{\vtheta,i}} + \frac{\lvert \rf_{n/2+i} \rvert^2}{\lambda_{\vtheta,n/2+i}} \right)\right]^{-1} =: 1-C_1 C_2^{-1}.
\end{align*}

For $0 \leq j < n/2$ we have 
\begin{align*}
    \frac{\partial L_\mathrm{KF}}{\partial \lambda_{\textcolor{red}{j}}} 
    =& \left[\frac{\lvert \rf_j + \rf_{n/2+j} \rvert^2}{(\lambda_{\vtheta,j} + \lambda_{\vtheta,n/2+j})^2}\right] C_2^{-1} - \frac{\lvert \rf_{\textcolor{red}{j}} \rvert^2}{\lambda_{\textcolor{red}{j}}^2} C_1 C_2^{-2} \\
    \frac{\partial L_\mathrm{KF}}{\partial \lambda_{\textcolor{red}{n/2+j}}}
    =& \left[\frac{\lvert \rf_j + \rf_{n/2+j} \rvert^2}{(\lambda_{\vtheta,j} + \lambda_{\vtheta,n/2+j})^2}\right]  C_2^{-1}  - \frac{\lvert \rf_{\textcolor{red}{n/2+j}} \rvert^2}{\lambda_{\textcolor{red}{n/2+j}}^2} C_1 C_2^{-2}
\end{align*}
where the only difference are the red indices.
Setting these equal to $0$ gives 
\begin{equation} \label{eq:KFoptcond}
\frac{\lambda_{\vtheta,{\textcolor{red}{j}}}}{\lambda_{\vtheta,j} + \lambda_{\vtheta,n/2+j}} = \frac{\lvert \rf_{\textcolor{red}{j}} \rvert \sqrt{C_1C_2}}{\lvert \rf_j + \rf_{n/2+j} \rvert}, \qquad 
\frac{\lambda_{\vtheta,{\textcolor{red}{n/2+j}}}}{\lambda_{\vtheta,j} + \lambda_{\vtheta,n/2+j}} = \frac{\lvert \rf_{\textcolor{red}{n/2+j}} \rvert \sqrt{C_1C_2}}{\lvert \rf_j + \rf_{n/2+j} \rvert},
\end{equation}
which implies that 
\[
1 = \frac{\lambda_{\vtheta,{j}} +  \lambda_{\vtheta,{n/2+j}}} {\lambda_{\vtheta,j} + \lambda_{\vtheta,n/2+j}} = 
\frac{(\lvert \rf_{j} \rvert + \lvert \rf_{n/2+j} \rvert) \sqrt{C_1C_2}}{\lvert \rf_j + \rf_{n/2+j} \rvert}
\]
and that 
\[
\frac{\lvert \rf_j + \rf_{n/2+j} \rvert} {\lvert \rf_{j} \rvert + \lvert \rf_{n/2+j} \rvert} = \sqrt{C_1C_2}
\]
should be constant for all $j$.  \textcolor{green}{[Not sure if this can be true.]}

But if it is true, then 
\begin{equation*}
\frac{\lambda_{\vtheta,{\textcolor{red}{j}}}}{\lambda_{\vtheta,j} + \lambda_{\vtheta,n/2+j}} = \frac{\lvert \rf_{\textcolor{red}{j}} \rvert }{\lvert \rf_{j} \rvert + \lvert \rf_{n/2+j} \rvert}, \qquad 
\frac{\lambda_{\vtheta,{\textcolor{red}{n/2+j}}}}{\lambda_{\vtheta,j} + \lambda_{\vtheta,n/2+j}} = \frac{\lvert \rf_{\textcolor{red}{n/2+j}} \rvert}{\lvert \rf_{j} \rvert + \lvert \rf_{n/2+j} \rvert},
\end{equation*}
and so
\begin{equation*}
\lambda_{\vtheta,{\textcolor{red}{j}}} =  c_j \lvert \rf_{\textcolor{red}{j}} \rvert, \qquad 
\lambda_{\vtheta,{\textcolor{red}{n/2+j}}}= c_j \lvert \rf_{\textcolor{red}{n/2+j}} \rvert,
\end{equation*}
which satisfies \eqref{eq:KFoptcond} for all $c_j$.  
Note that now $\lambda_j$ is proportional to $\rf_j$ and \emph{not its square}.
Moreover,
\begin{align*}
    C_1 & = \sum_{i=0}^{n/2-1} \frac{\lvert \rf_i + \rf_{n/2+i} \rvert^2}{\lambda_{\vtheta,i} + \lambda_{\vtheta,n/2+i}} 
    = \sum_{i=0}^{n/2-1} \frac{\lvert \rf_i + \rf_{n/2+i} \rvert^2}
    {c_i [\lvert \rf_i \rvert + \lvert \rf_{n/2+i} \rvert] } 
    = C_1C_2 \sum_{i=0}^{n/2-1} \frac{\lvert \rf_i \rvert + \lvert \rf_{n/2+i} \rvert}
    {c_i} \\
    C_2 &= \sum_{i=0}^{n/2-1} \left(\frac{\lvert \rf_i \rvert^2}{\lambda_{\vtheta,i}} + \frac{\lvert \rf_{n/2+i} \rvert^2}{\lambda_{\vtheta,n/2+i}} \right) 
    = \sum_{i=0}^{n/2-1} \frac{\lvert \rf_i \rvert + \lvert \rf_{n/2+i} \rvert} {c_i} 
\end{align*}

\textcolor{green}{[Still problematic.]}





\iffalse

For $0 \leq j < n/2$ we have 
\begin{align*}
    \frac{\partial L_\mathrm{KF}}{\partial \lambda_{\textcolor{red}{j}}} 
    =& \left[\frac{\lvert \rf_j + \rf_{n/2+j} \rvert^2}{(\lambda_j + \lambda_{n/2+j})^2}\right] \left[\sum_{i=0}^{n/2-1} \left(\frac{\lvert \rf_i \rvert^2}{\lambda_i} + \frac{\lvert \rf_{n/2+i} \rvert^2}{\lambda_{n/2+i}} \right)\right]^{-1} \\
    &- \frac{\lvert \rf_{\textcolor{red}{j}} \rvert^2}{\lambda_{\textcolor{red}{j}}^2} \left[\sum_{i=0}^{n/2-1} \frac{\lvert \rf_i + \rf_{n/2+i} \rvert^2}{\lambda_i + \lambda_{n/2+i}}\right] \left[\sum_{i=0}^{n/2-1} \left(\frac{\lvert \rf_i \rvert^2}{\lambda_i} + \frac{\lvert \rf_{n/2+i} \rvert^2}{\lambda_{n/2+i}} \right)\right]^{-2} \\
    \frac{\partial L_\mathrm{KF}}{\partial \lambda_{\textcolor{red}{n/2+j}}}
    =& \left[\frac{\lvert \rf_j + \rf_{n/2+j} \rvert^2}{(\lambda_j + \lambda_{n/2+j})^2}\right] \left[\sum_{i=0}^{n/2-1} \left(\frac{\lvert \rf_i \rvert^2}{\lambda_i} + \frac{\lvert \rf_{n/2+i} \rvert^2}{\lambda_{n/2+i}} \right)\right]^{-1} \\
    &- \frac{\lvert \rf_{\textcolor{red}{n/2+j}} \rvert^2}{\lambda_{\textcolor{red}{n/2+j}}^2} \left[\sum_{i=0}^{n/2-1} \frac{\lvert \rf_i + \rf_{n/2+i} \rvert^2}{\lambda_i + \lambda_{n/2+i}}\right] \left[\sum_{i=0}^{n/2-1} \left(\frac{\lvert \rf_i \rvert^2}{\lambda_i} + \frac{\lvert \rf_{n/2+i} \rvert^2}{\lambda_{n/2+i}} \right)\right]^{-2} 
\end{align*}
where the only difference are the red indices. Setting these equal to $0$ gives 
\begin{align*}
    \frac{\lvert \rf_{\textcolor{red}{j}} \rvert^2}{\lambda_{\textcolor{red}{j}}^2} 
    &= \frac{\lvert \rf_j + \rf_{n/2+j} \rvert^2}{(\lambda_j + \lambda_{n/2+j})^2} \cdot \frac{C_2}{C_1} \qquad\implies\qquad \lambda_{\textcolor{red}{j}} = \lvert \rf_{\textcolor{red}{j}} \rvert \frac{(\lambda_j + \lambda_{n/2+j})}{\lvert \rf_j + \rf_{n/2+j} \rvert} C \\
    \frac{\lvert \rf_{\textcolor{red}{n/2+j}} \rvert^2}{\lambda_{\textcolor{red}{n/2+j}}^2} 
    &= \frac{\lvert \rf_j + \rf_{n/2+j} \rvert^2}{(\lambda_j + \lambda_{n/2+j})^2} \cdot \frac{C_2}{C_1} \qquad\implies\qquad \lambda_{\textcolor{red}{n/2+j}} = \lvert \rf_{\textcolor{red}{n/2+j}} \rvert \frac{(\lambda_j + \lambda_{n/2+j})}{\lvert \rf_j + \rf_{n/2+j} \rvert} C.
\end{align*}
where $C = \sqrt{C_1/C_2}$.

For $x=\lambda_j$, $y=\lambda_{n/2+j}$, $\alpha = \frac{\lvert \rf_j \rvert}{\lvert \rf_j + \rf_{n/2+j} \rvert}C$, and $\beta = \frac{\lvert \rf_{n/2+j} \rvert}{\lvert \rf_j + \rf_{n/2+j} \rvert}C$ we would like to solve the system 
$$\begin{cases} x = \alpha (x+y) \\ y = \beta (x+y) \end{cases}.$$ 
This implies $x(1-\alpha) = \alpha y$ so $x = \alpha/(1-\alpha) y$ and 
$$y = \beta y \left(\frac{\alpha}{1-\alpha} + 1\right) = \frac{\beta}{1-\alpha} y$$
which implies $\lambda_j=\lambda_{n/2+j}=0$. \textcolor{blue}{Something must be off with the above derivation}

\fi


\bibliographystyle{amsplain}
\bibliography{FJH25,FJHown25,main}

%\end{document}

\section{Optimizing the Kernel According to Cross Validation}
\subsection{Initial optimization for the vertical scale factor}

Let $\ci \subseteq \{1,\dots,n\}$ be the included indices and denote the left out indices by $\overline{\ci} = \{1,\dots,n\} \setminus \ci$. Let us write the partitions
$$\mK = \begin{pmatrix} \mK_{\overline{\ci}\,\overline{\ci}}  & \mK_{\overline{\ci}\ci} \\ \mK_{\ci\overline{\ci}} & \mK_{\ci\ci} \end{pmatrix}, \qquad \mK^{-1} = \begin{pmatrix} \mA_{\overline{\ci}\,\overline{\ci}}  & \mA_{\overline{\ci}\ci} \\ \mA_{\ci\overline{\ci}} & \mA_{\ci\ci} \end{pmatrix}, \qquad \vf = \begin{pmatrix} \vf_{\overline{\ci}} \\ \vf_\ci \end{pmatrix}$$
and 
$$\vxi = \begin{pmatrix} \vxi_{\overline{\ci}} \\ \vxi_\ci\end{pmatrix} = \mK^{-1} \vf.$$
Then 
$$\check{\vf}_{\overline{\ci}} = \mK_{\overline{\ci} \ci} \mK_{\ci \ci}^{-1} \vf_\ci$$
and 
\begin{align*}
    \vxi_{\overline{\ci}} &= A_{\overline{\ci}\,\overline{\ci}} \vf_{\overline{\ci}} + A_{\overline{\ci}\ci} \vf_\ci \\
    &= A_{\overline{\ci}\,\overline{\ci}} \vf_{\overline{\ci}} - A_{\overline{\ci}\,\overline{\ci}} \mK_{\overline{\ci}\ci} \mK_{\ci \ci}^{-1} \vf_\ci \\
    &= A_{\overline{\ci}\,\overline{\ci}} \left(\vf_{\overline{\ci}} - \check{\vf}_{\overline{\ci}}\right) \\
    \implies \vf_{\overline{\ci}} - \check{\vf}_{\overline{\ci}} &= A_{\overline{\ci}\,\overline{\ci}}^{-1} \vxi_{\overline{\ci}}. 
\end{align*}
Let $\ci_1,\dots,\ci_F \subseteq \{1,\dots,n\}$ be mutually disjoint sets so that $\{1,\dots,n\} = \cup_{j=1}^F \ci_j$. Then the cross validation loss is 
$$\mathrm{CV} := \sum_{j=1}^F \left(\vf_{\overline{\ci_j}} - \check{\vf}_{\overline{\ci_j}}\right)^T \left(\vf_{\overline{\ci_j}} - \check{\vf}_{\overline{\ci_j}}\right) = \sum_{j=1}^F \vxi_{\overline{\ci_j}}^T A_{\overline{\ci_j}\,\overline{\ci_j}}^{-2} \vxi_{\overline{\ci_j}}.$$

\subsection{Leave One Out CV (LOOCV)} 

Leave one out CV sets $F=n$ and $\overline{\ci_j} = \{j\}$ for $j=1,\dots,n$. Then 
$$\mathrm{LOOCV} = \sum_{i=0}^{n-1} \frac{\xi_i^2}{a_{ii}^2}.$$
Since $\mK = \frac{1}{n} \mV \mLambda \mV^H$ is circulant, $\mK^{-1} = n \mV \mLambda^{-1} \mV^H$ is also circulant, so $\mK^{-1}$ has a constant diagonal, i.e. $a_{00} = a_{11} = \cdots$. Specifically, the first column of $\mK^{-1}$ is  
$$\va_0 = \frac{1}{n} \mV^H \mLambda^{-1} \boldsymbol{1},$$
so the constant along the diagonal is 
$$a_{00} = \frac{1}{n} \boldsymbol{1}^T \mLambda^{-1} \boldsymbol{1} = \frac{1}{n} \sum_{i=0}^{n-1} \lambda_i^{-1} = \boldsymbol{1}^T \vlambda^{-1}/n$$
where $\vlambda^{-1}$ is the elementwise inverse of the vector $\vlambda$. Therefore, 
$$\mathrm{LOOCV} = \frac{\sum_{i=0}^{n-1} \xi_i^2}{\left(\frac{1}{n} \sum_{i=0}^{n-1} \lambda_i^{-1}\right)^2} = n^6 \left(\sum_{i=0}^{n-1} \frac{\lvert \rf_i \rvert^2}{\lvert \lambda_i \rvert^2}\right) \left\lvert \sum_{i=0}^{n-1} \lambda_i^{-1}\right\rvert^{-2}$$
using 
$$\vxi^T \vxi = \vf^T \mK^{-1} \mK^{-1} \vf = n^2 \vf^T \mV \mLambda^{-2} \mV^H \vf = n^4 \rvf^H \mLambda^{-2} \rvf$$

\subsubsection{Optimization of $s^2$} 

\textcolor{blue}{Not sure what to do here as LOOCV is monotone in $s^2$.}

\subsubsection{Optimizing $\vtheta$} 

The derivative of LOOCV with respect to $\lambda_i$ is proportional to 

$$-2 \frac{\lvert \rf_i \rvert^2}{\lvert \lambda_i \rvert^{3}} \left\lvert \sum_{i=0}^{n-1} \lambda_i^{-1}\right\rvert^{-2} +2 \left(\sum_{i=0}^{n-1} \frac{\lvert \rf_i \rvert^2}{\lvert \lambda_i \rvert^2}\right) \left\lvert \sum_{i=0}^{n-1} \lambda_i^{-1}\right\rvert^{-3} \lambda_i^{-2}$$

\textcolor{blue}{Not sure where to go from here.} 

\section{GCV}

According to \cite{RatHic19a}, the generalized cross-validation choice of parameters defining the kernel is
\begin{align*}
    \vtheta_{\text{GCV}} 
    &= \argmin_{\vtheta} \left [\log\left( \sum_{i=0}^{n-1} \frac{\lvert\rf_i\rvert^2}{\lambda_{\vtheta,i}^2} \right)
    - 2 \log \left(\sum_{i=0}^{n-1} \frac{1}{\lambda_{\vtheta,i}} \right) \right ]
\end{align*}

Again, suppose for a moment that the $\lambda_{\vtheta}$ can be chosen arbitrarily.  Dropping the $\vtheta$ dependence, we want to satisfy
\begin{gather*}
    \begin{cases}
    \displaystyle \frac {2}{\lambda_0^2} \left( \sum_{j=0}^{n-1} \frac{1}{\lambda_{j}} \right)^{-1}  = 0, \\ 
    \displaystyle - \frac{2\lvert\rf_i\rvert^2}{\lambda_{i}^3} \left( \sum_{j=1}^{n-1} \frac{\lvert\rf_j\rvert^2}{\lambda_{j}^2} \right)^{-1} +  \frac{2}{\lambda_i^2} \left( \sum_{j=0}^{n-1} \frac{1}{\lambda_{j}} \right)^{-1} = 0, \quad i = 1, \ldots, n-1
    \end{cases}
    \\
    \iff
    \begin{cases}
    \lambda_0 = \infty,  \\ 
    \displaystyle \frac{\lvert\rf_i\rvert^2}{\lambda_{i}} =  \left( \sum_{j=1}^{n-1} \frac{\lvert\rf_j\rvert^2}{\lambda_{j}^2} \right) \, \left( \sum_{j=1}^{n-1} \frac{1}{\lambda_{j}} \right)^{-1} = 0, \quad i = 1, \ldots, n-1
    \end{cases}
    \\
    \iff
     \lambda_0 = \infty, \qquad \lambda_i = c \lvert\rf_i\rvert^2, \quad i = 1, \ldots, n-1.
\end{gather*}

\end{document}