\documentclass{amsart}
\usepackage{mathtools,upref,siunitx,upquote,fancyvrb,xspace,color}
\usepackage[hyphens]{url}
\usepackage[utf8]{inputenc}
\usepackage{esdiff}

\input{FJHDef.tex}


\usepackage{algpseudocode}
\usepackage{algorithm, algorithmicx}
\algnewcommand\algorithmicparam{\textbf{Parameters:}}
\algnewcommand\PARAM{\item[\algorithmicparam]}
\algnewcommand\algorithmicinput{\textbf{Input:}}
\algnewcommand\INPUT{\item[\algorithmicinput]}
\algnewcommand\RETURN{\State \textbf{Return }}



\begin{document}
\title{Sample Document}
\author{Fred J. Hickernell}
\author{Houman Owhadi}
\author{Aleksei Sorokin}
\begin{abstract}This project is where all of the files and commands go that are needed elsewhere
\end{abstract}

\maketitle

\section{Introduction}

\section{Background}
Let $f$ have an absolutely summable Fourier series
\begin{equation*}
    f(\vx) = \sum_{\vk \in \integers^d} \hf(\vk) \exp(2 \pi \sqrt{-1} \vk^T \vx), \qquad \hf(\vk) : = \int_{[0,1]^d} f(\vx)  \exp(-2 \pi \sqrt{-1} \vk^T \vx) \, \dif \vx.
\end{equation*}
Suppose that $\{ \vx_1, \ldots, \vx_n\}$ is a shifted integration lattice, 
\begin{equation*}
    \vx_i = \vh i /n + \vDelta \pmod \vone, \qquad i = 0, \ldots, n - 1
\end{equation*}
where $\vh \in \{1, \ldots, n-1\}^d$.  Then the DFT of the function data, $\{ f(\vx_0), \ldots, f(\vx_{n)-1}\}$, is given by 
\begin{align*}
    \tf(\vk) &= \frac 1n \sum_{i=1}^n f(\vx_i) \exp(-2 \pi \sqrt{-1} \vk^T \vx_i) \\
    & = \frac 1n \sum_{i=1}^n f(\vx_i) \exp(-2 \pi \sqrt{-1} [\vk^T\vh i/n + \vk^T\vDelta]) \\
    & = \rf(\vk^T \vh \bmod n) \exp(-2 \pi \sqrt{-1} \vk^T\vDelta), \\
    \text{where } 
    \rf(j) &: = \frac 1n \sum_{i=1}^n f(\vx_i) \exp(-2 \pi \sqrt{-1} ij/n), \quad  j = 0, \ldots, n-1 \\
    & = \frac 1n \sum_{i=1}^n \sum_{\vk \in \integers^d} \hf(\vk) \exp(2 \pi \sqrt{-1} \vk^T \vx_i) \exp(-2 \pi \sqrt{-1} ij/n) \\
    & = \frac 1n \sum_{\vk \in \integers^d} \hf(\vk) \sum_{i=1}^n \exp(2 \pi \sqrt{-1} i (\vk^T \vh - j) /n + \vk^T \vDelta) \\
    & = \sum_{\substack{\vk \in \integers^d \\ \vk^T \vh \bmod n = j}} \hf(\vk) \exp(2 \pi \sqrt{-1} \vk^T \vDelta) \\
\end{align*}
Note that the DFT takes on only $n$ distinct values. 

Now consider a periodic kernel:
\begin{equation*}
    K(\vt,\vx) = \sum_{\vk \in \integers^d} \omega_{\vk}^2 \exp(2 \pi \sqrt{-1} \vk^T (\vt - \vx))
\end{equation*}
where the set of weights, $\vomega : = (\omega_{\vk})_{\vk \in \integers^d}$, satisfies the following summability condition:
\begin{equation*}
    \sum_{\vk \in \integers^d} \omega_{\vk}^2 < \infty.
\end{equation*}
We assume that $\omega_{-\vk} = \omega_{\vk}$ For such a kernel, the inner product of the associated reproducing kernel Hilbert space is defined by 
\begin{equation*}
    \ip[\vomega]{f}{g} = \sum_{\vk \in \integers^d} 
    \frac{\overline{\hf(\vk)} \hg(\vk)}{\omega_{\vk}^2}
\end{equation*}
The reproducing property follows because $K(\cdot,\vx)$ is in the Hilbert space, and 
\begin{equation*}
    \ip[\vomega]{K(\cdot,\vx)}{f} 
    = \sum_{\vk \in \integers^d} 
    \frac{ \omega_{\vk}^2 \exp(2 \pi \sqrt{-1} \vk^T \vx)) \hf(\vk)}{\omega_{\vk}^2} = f(\vx).
\end{equation*}

Now consider the entries of the Gram matrix for a periodic kernel:
\begin{align*}
    K(\vx_i,\vx_j)  &= \sum_{\vk \in \integers^d} \omega_{\vk}^2 \exp(2 \pi \sqrt{-1} \vk^T(\vx_i - \vx_j)) \\
    & = \sum_{\vk \in \integers^d} \omega_{\vk}^2 \exp(2 \pi \sqrt{-1} \vk^T \vh (i-j)/n) \\
    & = \sum_{l = 0}^{n-1} \lambda_l \exp(2 \pi \sqrt{-1} l (i-j)/n), \quad \lambda_l =: \sum_{\substack{\vk \in \integers^d \\ \vk^T\vh \bmod n = l}} \omega_{\vk}^2 \\
    \MoveEqLeft \underbrace{\bigl ( K(\vx_i,\vx_j)\bigr)_{i,j = 0}^{n-1}}_{\mK} \\
    &
    = \underbrace{\bigl ( \exp(-2 \pi \sqrt{-1} jl/n) \bigr)_{j,l = 0}^{n-1}}_{\mV^H} 
    \underbrace{\bigl ( \lambda_l \delta_{l,m} \bigr)_{l,m = 0}^{n-1}}_{\diag(\vlambda)} 
    \underbrace{\bigl ( \exp(2 \pi \sqrt{-1} mi/n) \bigr)_{m,i = 0}^{n-1}}_{\mV}
\end{align*}
Moreover, not that 

\section{Optimizing the Kernel according to Empirical Bayes}



\bibliographystyle{amsplain}
\bibliography{FJH25,FJHown25}

\end{document}